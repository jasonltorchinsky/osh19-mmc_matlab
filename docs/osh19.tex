The Ogrosky--Stechmann--Hottovy 2019 (OSH19) system, introduced in \cite{Ogrosky19}, is a three-dimensional, non-linear, deterministic tropical channel model for the Madden-Julian Oscillation (MJO) and convectively-coupled equatorial waves (CCEWs). The model consists of the moist hydrostatic incompressible Boussinesq equations with a 2-vertical-mode convective adjustment and moisture diffusion. As in \cite{Ogrosky19}, we give the dynamical equations here (Eqs.~\ref{eqs:osh19}) in dimensional form in their original formulation.

\begin{subequations}
	\begin{align}
		\mdv{u} - \beta\,y\,v + \pdv{p}{x} &= -\frac{1}{\tau_u}\,u, \label{eqn:zonal_wind} \\
		\mdv{v} + \beta\,y\,u + \pdv{p}{y} &= -\frac{1}{\tau_u}\,v, \label{eqn:merid_wind} \\
		\pdv{p}{z} &= \frac{g}{\theta_0}\,\theta, \label{eqn:hydrostatic} \\
		\pdv{u}{x} + \pdv{v}{y} + \pdv{w}{z} &= 0, \label{eqn:incompressible} \\
		\mdv{\theta} + B\,w &= \frac{L_v}{c_p\,\tau_{\text{up}}}\,q_{\text{up}} + \frac{L_v}{c_p\,\tau_{\text{mid}}}\,q_{\text{mid}} - \frac{1}{\tau_{\theta}}\,\theta, \label{eqn:therm} \\
		\mdv{q} + v\,\pdv{q_{\text{bg}}}{y} + w\,\pdv{q_{\text{bg}}}{z} &= -\frac{1}{\tau_{\text{up}}}\,q_{\text{up}} - \frac{1}{\tau_{\text{mid}}}\,q_{\text{mid}} + \func{D_h}{z}\,\gpr{\pdv[2]{q}{x} + \pdv[2]{q}{y}} + D_v\,\pdv[2]{q}{z}, \label{eqn:moist}
	\end{align}
	\label{eqs:osh19}
\end{subequations}

where $\mdv{} = \partial_t + u\,\partial_x + v\,\partial_y + w\,\partial_z$ is the material derivative and $\va{u} = \mqty[u & v & w]^T$ is the vector of zonal, meridional, and vertical velocity components of winds. The variables $p$, $\theta$, and $q$ represent anomalies of pressure, potential temperature, and moisture, respectively, from a background state. Meanwhile, $q_{\text{up}}$ and $q_{\text{mid}}$ are convective adjustment terms. We shall elaborate on the form of the background states as well as the convective adjustment terms below.

The convective adjustment scheme consists of two vertical modes, constructed from the first two baroclinic modes fro moisture. Specifically, the baroclinic modes for moisture are given by

\begin{equation}
	\func{S_k}{z} = \sqrt{2}\,\func{\sin}{k\,\gpr{\frac{\pi\,z}{H}}},\qquad k = 1,\ 2,\ \dots,\ 
\end{equation}

where the domain extends from the planetary surface $z = 0$ to the top of the troposphere $z = H$. We define the mid- and upper-baroclinic modes of moisture using only the first two baroclinic modes for moisture

\begin{subequations}
	\begin{align}
		\func{S_{\text{low}}}{z} &= \frac{1}{\sqrt{2}}\,\gpr{\func{S_1}{z} + \func{S_2}{z}}, \\
		\func{S_{\text{up}}}{z} &= \frac{1}{\sqrt{2}}\,\gpr{\func{S_1}{z} - \func{S_2}{z}},
	\end{align}
\end{subequations}

so that $S_{\text{low}}$, $S_{\text{up}}$, and $S_k$ for $k = 3,\ 4,\ \dots$ form an orthonormal basis in the vertical domain with norm

\begin{equation}
	\gang{\func{f}{z},\ \func{g}{z}} = \frac{1}{H}\,\int_{0}^{H} \func{f}{z}\,\func{g}{z}\,dz.
\end{equation}

Therefore, we define the lower- and upper-vertical modes of moisture to be

\begin{subequations}
	\begin{align}
		\func{q_{\text{low}}}{t,\ \va{x}_h} &= \frac{1}{H}\,\int_{0}^{H} \func{q}{t,\ \va{x}}\,\func{S_{\text{low}}}{z}\,dz, \\
		\func{q_{\text{up}}}{t,\ \va{x}_h} &= \frac{1}{H}\,\int_{0}^{H} \func{q}{t,\ \va{x}}\,\func{S_{\text{up}}}{z}\,dz.
	\end{align}
\end{subequations}


Also, $p$ is pressure anomalies from a background state, $\theta = \func{\theta}{t,\ \va{x}} = \func{\theta}{t,\ x,\ y,\ z}$ is potential temperature anomalies from a background state $\func{\theta_{\text{BG}}}{z} = \theta_0 + B\,z$, and the moisture variable $q = \func{q}{t,\ \va{x}}$ represents specific hulowity anomalies from a background state $\func{q_{\text{BG}}}{y,\ z} = q_0 + \func{q_{\text{bg}}}{y,\ z}$, where the background mositure anomaly $\func{q_{\text{bg}}}{y,\ z}$ is given by

\begin{equation}
	\func{q_{\text{bg}}}{y,\ z} = \gpr{1 - a\,\gpr{1 - e^{-\frac{y^2}{2\,\widetilde{L}^2}}}}\,\func{q_{\text{bg},\text{eq}}}{z},
\end{equation}

where $\func{q_{\text{bg},\text{eq}}}{z} = B_{vs}\,\gpr{z - H}$ is a prescribed vertial profile of background moisture at the equator, $0 \leq a \leq 1$ and $\widetilde{L}$ dictate the background moisture meridional decay away from the equator, and $B_{vs}$ is the vertical gradient of background moisture. The potential temperature and moisture at $z = 0$ (the top of the boundary layer) are given by $\theta_0$ and $q_0$, respectively. The terms propotional to $q_{\text{up}}$ and $q_{\text{low}}$ in Eqns.~\ref{eqn:therm} and \ref{eqn:moist} are convective adjustment terms. In the 2-vertical-mode scheme used here, they are defined using only the first two baroclinic modes of moisture

\begin{subequations}
	\begin{align}
		\func{q_{\text{up}}}{t,\ \va{x}} &= \sqrt{2}\,Q_{\text{up}}\,\gpr{\func{\sin}{\frac{\pi\,z}{H}} - \frac{1}{2}\,\func{\sin}{\frac{2\,\pi\,z}{H}}},\\[6pt]
		\func{q_{\text{low}}}{t,\ \va{x}} &= \sqrt{2}\,Q_{\text{low}}\,\func{\sin}{\frac{\pi\,z}{H}},
	\end{align}
\end{subequations}

where $H$ is the height of the troposphere and 

\begin{subequations}
	\begin{align}
		Q_{\text{up}} &= \sqrt{2}\,q_1\,\func{\sin}{\frac{2\,\pi}{3}} + 2\,\sqrt{2}\,q_2\,\func{\sin}{\frac{4\,\pi}{3}},\\[6pt]
		Q_{\text{low}} &= \sqrt{2}\,q_1\,\func{\sin}{\frac{\pi}{2}}.
	\end{align}
\end{subequations}

The terms $\func{q_1}{t,\ \va{x}_{h}} = \func{q_1}{t,\ x,\ y}$ and $\func{q_2}{t,\ \va{x}_h}$ represent the first and second baroclinic components of moisture, respectively, and are given by

\begin{subequations}
	\begin{align}
		\func{q_{1}}{t,\ \va{x}_{h}} &= \frac{\sqrt{2}}{H}\,\int_{0}^{H} \func{q}{t,\ \va{x}}\,\func{\sin}{\frac{\pi\,z}{H}}\,dz,\\[6pt]
		\func{q_{2}}{t,\ \va{x}_{h}} &= \frac{2\,\sqrt{2}}{H}\,\int_{0}^{H} \func{q}{t,\ \va{x}}\,\func{\sin}{\frac{2\,\pi\,z}{H}}\,dz.
	\end{align}
\end{subequations}

The diffusion coefficient $\func{D_h}{z} = D_{h,\text{low}} + \frac{D_{h,\text{up}}}{H}\,z$ is a linear intrpolation of the lower- and upper-troposphere diffusion coefficients.

The parameters for the ``standard'' case are given in Table~\ref{tbl:std_params}.

\begin{table}[H]
	\centering
	\begin{tabular*}{\textwidth}{l @{\extracolsep{\fill}}l @{\extracolsep{\fill}}l}
		\hline
		Parameter & Description & Standard Value \\ \hline
		$H$ & Height of troposphere & $16\ km$ \\
		$p_Y$ & Distance from equator to channel wall & $6000\ km$ \\
		$\beta$ & Variation of Coriolis parameter with latitude & $2.3 \times 10^{-11}\ m^{-1}\,s^{-1}$ \\
		$g$ & Acceleration due to gravity & $9.8\ m\,s^{-2}$ \\
		$c_p$ & Specific heat of dry air at constant pressure & $1006\ J\,kg^{-1}\,K^{-1}$ \\
		$L_v$ & Latent heat of vaporization & $2.5 \times 10^{6}\ J\,kg^{-1}$ \\
		$B$ & Background potential temperature vertical gradient & $3\ K\,km^{-1}$ \\
		$\theta_0$ & Reference potential temperature & $300\ K$ \\
		$\tau_u$ & Wind damping time-scale & $25\ d$ \\
		$\tau_{\theta}$ & Potential temperature damping time-scale & $25\ d$ \\
		$\tau_{\text{up}}$ & Moisture damping time-scale (upper-troposphere) & $1\ d$ \\
		$\tau_{\text{low}}$ & Moisture damping time-scale (low-troposphere) & $1/12\ d$ \\
		$B_{vs}$ & Mean vertical $q_{\text{BG}}$ gradient & $-1.34 \times 10^{-3}\ kg\ kg^{-1}\ km^{-1}$ \\
		$a$ & 1 - pole--to--equator $q_{\text{BG}}$ ratio & $0.25$ \\
		$\widetilde{L}$ & $q_{\text{BG}}$ meridional decay length-scale & $2000\ km$ \\
		$D_{h,up}$ & Horizontal $q$ diffusion (upper-troposphere) & $60.8\ km^2\,s^{-1}$ \\
		$D_{h,low}$ & Horizontal $q$ diffusion (lower-troposphere) & $7.6\ km^2\,s^{-1}$ \\
		$D_v$ & Vertical $q$ diffusion & $10^{-4}\ km^2\,s^{-1}$ \\ \hline
	\end{tabular*}
	\caption{Physical parameter values for the ``standard'' case.} 
	\label{tbl:std_params}
\end{table}